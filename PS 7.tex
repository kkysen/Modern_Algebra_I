\documentclass[fleqn]{article}
\usepackage[margin=1in]{geometry}
\usepackage[utf8]{inputenc}
\usepackage{ulem}
\usepackage{mathtools}
\usepackage{amsmath}
\usepackage{amsthm}
\usepackage{amssymb}
\usepackage{fancyvrb}
\usepackage{cleveref}
\usepackage{centernot}

\title{
Math GU4041, Fall 2019 \\
Modern Algebra I, Prof.\ Kyler Siegel \\
Problem Set 7
}
\author{Khyber Sen}
\date{11/8/2019}

\DeclareMathOperator{\im}{im}
\DeclareMathOperator{\orb}{orb}
\DeclareMathOperator{\stab}{stab}
\DeclareMathOperator{\Cl}{Cl}
\DeclareMathOperator{\cycletype}{cycle\_type}

\setcounter{secnumdepth}{0}

\begin{document}
    
    \maketitle
    
    \section{Problem 1}
    Prove that every group of order 35 is cyclic.  You may invoke general results such as Lagrange's Theorem and Sylow's Theorems but not any specific classification results for groups of given orders.
        
        \subsection{Solution}
        Let $p = 5$ and $q = 7$.  Then $|G| = pq$, so $p \mid |G|$ and $g \mid |G|$.  Thus, by Sylow's 3rd Theorem, $n_p \equiv 1$ (mod $p$) and $n_p \mid q$.  Since $p \nmid q - 1$, $n_p$ must be 1.  Applying Sylow's 3rd Theorem to $q$, we get $n_q \equiv 1$ (mod $q$) and $n_q \mid p$.  Since $p < q$, $n_q$ must be 1.  Since $n_p$ and $n_q$ are both 1, we can name these subgroups.  Let $P$ be the unique Sylow $p$-subgroup and $Q$ be the unique Sylow $q$-subgroup.  The order of every element of $P$ must divide $p$, so the order of every element of $P$ must be either 1 or $p$.  Only the identity has order 1, so the remaining $p - 1$ elements have order $p$.  The same applies to $G$.
        
        Assume $G$ is not cyclic.  Then $G$ has no elements of order $|G| = pq$.  Thus, every element of $G$ has order $1$, $p$, or $q$, since $pq$ is ruled out.  $P$ contains all the elements of order $p$, $Q$ contains all the elements of order $q$, and they both share $e$ of order 1.  Thus, there should be $p + q - 1$ elements in $G$ if there are no elements of order $pq$.  However, $p + q - 1 \neq pq$, a contradiction, meaning $G$ must be cyclic.
    
    \section{Problem 2}
    Suppose that $p$ and $q$ are prime numbers with $p < q$ such that $p$ divides $q - 1$.  The goal of this problem is to prove that there is a non-abelian group of order $pq$.
        
        \subsection{(1)}
        Prove that there is a subgroup $Q$ of $S_q$ of order $q$.
            
            \subsubsection{Solution}
            Let $\sigma_n = (1..n)$, i.e. a $n$-cycle.  The group generated by an $n$-cycle, $\langle \sigma_n \rangle$, is a cyclic subgroup since it is generated, and it has order $n$ since it takes $n$ times for the $n$-cycle to completely cycle around to the identity.  Thus, if we let $Q = \langle \sigma_q \rangle$, then $|Q| = q$ and $Q \subseteq S_q$.
        
        \pagebreak
        
        \subsection{(2)}
        Prove that the normalizer $N_{S_q}(Q)$ has a subgroup $P$ of order $p$.
            
            \subsubsection{Solution}
            In $S_q$, elements of order $q$ are $q$-cycles.  These $q$-cycles have the form $(1, n_2, ..., n_q)$, meaning there are $(q - 1)!$ $q$-cycles, and thus $(q - 1)!$ elements of order $q$.  Since each subgroup generated by a $q$-cycle has $q - 1$ elements of order $q$ (the subgroups are of order $q$ and have $q$ elements, but $e$ has order 1), there are $\frac{(q - 1)!}{q - 1} = (q - 2)!$ subgroups of order $q$.  Thus, $n_q = (q - 2)!$.  We also know $n_q = [S_q : N_{S_q}(Q)]$.  Thus, $(q - 2)! = \frac{|S_q|}{|N_{S_q}(Q)|} = \frac{q!}{|N_{S_q}(Q)|}$.  Therefore, $|N_{S_q}(Q)| = \frac{q!}{(q - 2)!} = q(q - 1)$.  
            
            Since $p$ divides $q - 1$, $p$ divides $q(q - 1)$ and thus $p$ divides $|N_{S_q}(Q)|$, meaning by Sylow's 1st Theorem, there exists an order $p$ Sylow $p$-subgroup of $N_{S_q}(Q)$.
        
        \subsection{(3)}
        Prove that $PQ$ is a group of order $pq$.
            
            \subsubsection{Solution}
            $|PQ| = \frac{|P||Q|}{|P \cap Q|} = \frac{pq}{|P \cap Q|}$.  Thus, we just need to show $|P \cap Q| = 1$.  $|PG|$ is a whole number, so $|P \cap Q|$ must divide $|P||Q|$, so $|P \cap Q| \in \{1, p, q\}$.  $q$ is obviously impossible since $|P| < |Q|$ and $|P \cap Q| \leq \min(|P|, |Q|)$.  $p$ is also impossible, because if $|P \cap Q| = p$, then $P \cap Q = P$, meaning $P \subseteq Q$.  But then $p$ would divide $q$ by Lagrange's Theorem, but $p$ and $q$ are primes, so this is a contradiction.  Therefore, $|P \cap Q| = 1$, the only option left.  Thus, $|PQ| = \frac{|P||Q|}{|P \cap Q|} = \frac{pq}{1} = pq$.
        
        \subsection{(4)}
        Prove that $PQ$ is not abelian.
            
            \subsubsection{Solution}
            $P$ and $Q$ are of prime order, meaning they're cyclic and $P \cong \mathbb{Z}/p\mathbb{Z}$ and $Q \cong \mathbb{Z}/q\mathbb{Z}$.  Furthermore,
            \begin{align}
                \mathbb{Z}/p\mathbb{Z} &\cong P' :=  \left\{\begin{bmatrix}
                    p & 0 \\ 
                    0 & 1
                \end{bmatrix} \thickspace \middle\vert \thickspace p \in \mathbb{Z}/p\mathbb{Z}\right\} \\
                \mathbb{Z}/q\mathbb{Z} &\cong Q' :=  \left\{\begin{bmatrix}
                    1 & 0 \\ 
                    q & 1
                \end{bmatrix} \thickspace \middle\vert \thickspace q \in \mathbb{Z}/q\mathbb{Z}\right\} \\
                PQ &\cong P'Q' \\
                    &= \left\{\begin{bmatrix}
                        p & 0 \\ 
                        0 & 1
                    \end{bmatrix}
                    \begin{bmatrix}
                        1 & 0 \\ 
                        q & 1
                    \end{bmatrix} \thickspace \middle\vert \thickspace p \in \mathbb{Z}/p\mathbb{Z}, q \in \mathbb{Z}/q\mathbb{Z}\right\} \\
                    &= \left\{\begin{bmatrix}
                        p & 0 \\ 
                        q & 1
                    \end{bmatrix} \thickspace \middle\vert \thickspace p \in \mathbb{Z}/p\mathbb{Z}, q \in \mathbb{Z}/q\mathbb{Z}\right\}
            \end{align}
            Now we just have to show that this is non-abelian, which is simple:
            \begin{align}
                \begin{bmatrix}
                    a & 0 \\ 
                    b & 1
                \end{bmatrix} \begin{bmatrix}
                    c & 0 \\ 
                    d & 1
                \end{bmatrix} &= \begin{bmatrix}
                    ac & 0 \\ 
                    bc + d & 1
                \end{bmatrix} \\
                \begin{bmatrix}
                    c & 0 \\ 
                    d & 1
                \end{bmatrix} \begin{bmatrix}
                    a & 0 \\ 
                    b & 1
                \end{bmatrix} &= \begin{bmatrix}
                    ca & 0 \\ 
                    da + b & 1
                \end{bmatrix}
            \end{align}
            Although $ac = ca$, $bc + d \neq ad + b$, so this isn't abelian.
    
    \section{Problem 3}
    Let $G$ be a finite simple group with $|G| < 100$.  Prove that $G$ is either abelian or has order 60. \\
    \textit{Hints:
        \begin{itemize}
            \item Count the elements of $G$ in terms of their orders.  When does this exceed $|G|$?
            \item If $H$ is a large subgroup of $G$, consider the action of $G$ on the set of the left cosets of $H$ in $G$.  What does the first isomorphism theorem say?
            \item Similarly, if $H$ is a Sylow subgroup of $G$, consider the action of $G$ on the set of conjugates in $H$.
        \end{itemize}
    }
    
        \subsection{Solution}
        Let's try to classify all the finite simple groups of order $< 100$.  
        
        First, $\{e\}$ is trivially simple and abelian.  
        
        Second, groups of prime order are cyclic, and prime order cyclic groups have only the trivial subgroup and themselves as subgroups, so they cannot have any nontrivial proper normal subgroups, and thus are simple.  Since they are cyclic, they are also abelian.  
        
        Third, groups of prime power order, $p$-groups, have a nontrivial center (proved in pset \#6, problem \#2).  Since the center is a normal subgroup, since they have nontrivial centers, they cannot be simple.
        
        Fourth, if $|G| = p^a m$, where $p$ is prime and $m < p$, then we can show $G$ is not simple.
        
        Thus, we are left with the following orders of $G$ that we have to show are either abelian, of order 60, or not simple: 12, 24, 30, 36, 40, 45, 48, 56, 60, 63, 70, 72, 80, 84, 90, 96.
        
        Now we can apply Sylow's Theorems to start eliminating orders.  If $n_p = 1$ for any prime factor $p$, then the unique Sylow $p$-subgroup is normal (b/c Sylow $p$-subgroups are conjugate are there is only 1), meaning $G$ isn't simple.
        
        For $|G| = 5a$, $n_5 \equiv 1$ (mod 5) and $n_5 \mid a$.  For $a =$ 8, 9, 14, 18, this implies $n_5 = 1$.  Thus, we can eliminate $|G| =$ 40, 45, 70, 90, leaving 12, 24, 30, 36, 48, 56, 60, 63, 72, 80, 84, 96.
        
        For $|G| = 30$, $n_5 \in \{1, 6\}$ and $n_3 \equiv 1$ (mod 3) and $n_3 \mid 10$, so $n_3 \in \{1, 30\}$.  Assuming neither $n_p = 1$, $n_5 = 6$ and $n_3 = 10$.  Counting elements of order 5 and 3, we get $6(5 - 1) + 10(3 - 1) = 42 > 30$, which can't be true.  Thus, one $n_p = 1$, so $G$ isn't simple.
        
        For $|G| = 80$, $n_5 \in \{1, 16\}$ and $n_2 \equiv 1$ (mod 2) and $n_2 \mid 5$, so $n_2 \in \{1, 5\}$.  Assuming $n_p \neq 1$, $n_5 = 16$ and $n_2 = 5$.  Counting elements, we get at least $16(5 - 1) + 5(16 - 1) = 139 > 80$, so $G$ can't be simple.
        
        For $|G| = 7a$, $n_7 \equiv 1$ (mod 7) and $n_7 \mid a$.  For $a =$ 9, 12, this implies $n_7 = 1$, so $G$ isn't simple.  For $a = 8$, $n_7 \in \{1, 8\}$.  Assuming all $n_p \neq 1$, $n_7 = 8$.  Then $n_2 \equiv 1$ (mod 2) and $n_2 \mid 7$.  Thus, $n_2 \in \{1, 7\}$, so $n_2 = 7$.  Counting elements, we get at least $8(7 - 1) + 7(8 - 1) = 97 > 56$, so $G$ can't be simple.  This eliminates $|G| =$ 56, 63, 84, and 30 and 80 from before, leaving 12, 24, 36, 48, 60, 72, 96.
        
        For $|G| = 2^a 3^b$ where $a \in 2..5$ and $b \in 1..2$, $n_3 \equiv 1$ (mod 3) and $n_3 \mid 2^a$, and $n_2 \equiv 1$ (mod 2) and $n_2 \mid 3^b$.  If we choose the minimal $n_p \neq 1$ for these, we get $n_3 = 4$ and $n_2 = 3$.  Thus, we need to check if $4(3^b - 1) + 3(2^a - 1) > 2^a 3^b \implies 4(3^b) + 3(2^a) > 2^a 3^b + 7$.  If $b = 1$, then this simplifies to $4(3) + 3(2^a) > 3(2^a) + 7 \implies 12 > 7$, which is true.  If $a = 2$, then this simplifies to $4(3^b) + 3(4) > 4(3^b) + 7 \implies 12 > 7$, which is true.  These cases cover most of the remaining orders, including $12 = 2^2 \cdot 3^1$, $24 = 2^3 \cdot 3^1$, $36 = 2^2 \cdot 3^2$, $48 = 2^4 \cdot 3^1$, $96 = 2^5 \cdot 3^1$.  Thus, only 60 and 72 remain.
        
        For $|G| = 72$, we know if $n_p \neq 1$, then $n_2 = 3$ and $n_3 = 4$.  Counting elements doesn't work here as seen above.  Instead, consider the conjugation action of $G$ on the 4 Sylow 3-subgroups.  This action has a corresponding homomorphism $f: G \to S_4$.  By the 1st Isomorphism Theorem, $G/\ker(f) \cong \im(f) \subseteq S_4$.  Thus, by Lagrange's Theorem, $|G/\ker(f)| = \frac{72}{|\ker(f)|} = \frac{2^3 \cdot 3^2}{|\ker(f)|}$ divides $S_4 = 4! = 24 = 2^3 \cdot 3$.  Thus, we have $\frac{2^3 \cdot 3^2}{|\ker(f)|} \mid 2^3 \cdot 3^1$.  Thus, $|\ker(f)| = 3$.  Since $\ker(f) \unlhd G$ and $|\ker(f)| \notin \{1, |G|\}$, $G$ can't be simple.
        
        \pagebreak
        
        This just leaves $|G| = 60$.  $G$ might be simple, since $A_5$ has order 60 and is simple, so we need to prove $G$ can't be abelian.  In particular, $G \cong A_5$, which is non-abelian.  Proving $G \cong A_5$ is more complicated, though.  It's much easier to just prove $G$ is non-abelian.  
        
        Assume $G$ is abelian.  Then every subgroup is normal.  But since $G$ is simple, there are no nontrivial proper normal subgroups.  Thus, there can't be any nontrivial proper subgroups.  By Sylow's 1st Theorem, $G$ must have a Sylow 2, 3, and 5-subgroup, which are nontrivial proper subgroups, a contradiction.  Thus $G$ can't be abelian.
        
        Thus, we've classified the finite simple groups of order $< 100$ into cyclic groups of prime order, which are abelian, and the non-abelian $A_5$ of order 60.
    
    \section{Problem 4}
    Prove that a group of order 1365 cannot be simple.
        
        \subsection{Solution}
        Let $G$ be a group of order 1365.  The prime factorization of 1365 is $1365 = 3 \cdot  5\cdot 7 \cdot 13$.  If any of $n_p$, where $p$ is a prime factor of 1365, then the unique Sylow $p$-subgroup $P$ must be normal (because Sylow $p$-subgroups are conjugates of each other, and if $P$ is the unique one, then $gPg^{-1} = P$ $\forall$ $g \in G$, meaning $P$ is normal).  By Sylow's 3rd Theorem, we can place restrictions on $n_p$: $n_p \equiv 1$ (mod $p$) and $n_p \mid \frac{|G|}{p^k} = \frac{|G|}{p}$ (since the prime factorization of 1365 has unique primes).
        
        Thus, $n_3 \in \{1, 7, 13, 91\}$, $n_5 \in \{1, 21, 91\}$, $n_7 \in \{1, 15\}$, and $n_{13} \in \{1, 105\}$.  If any $n_p = 1$, then we're done, since the nontrivial, proper Sylow $p$-subgroup is normal, so assume all the $n_p \neq 1$.  Thus, $n_7 = 15$ and $n_{13} = 105$.  If we choose the minimum values for $n_3$ and $n_5$, we get $n_3 = 7$ and $n_5 = 21$.  Thus, $|G| > 105(13 - 1) + 15(7 - 1) + 21(5 - 1) + 7(3 - 1) = 1448 \not< 1365$.  This is a contradiction.  Thus, at least one $n_p$ must be 1, so there is a nontrivial proper normal subgroup, meaning $G$ can't be simple.
    
    \section{Problem 5}
    For which $n \in \mathbb{Z}^+$ does there exists a group $G$ of order $n$ with precisely two conjugacy classes?
        
        \subsection{Solution}
        $e$ is always in the center $Z(G)$, meaning it's always in its own singleton conjugacy class.  Thus, for $G$ to have exactly two conjugacy classes, every non-identity element must be in the second conjugacy class.  Thus, if we consider the conjugation action of $G$ on itself, we need $\orb(a) = G \setminus \{e\}$, where $a \neq e$.  $|\orb(a)| = |G \setminus \{e\}| = |G| - 1 = n - 1$.  By the Orbit-Stabilizer Theorem, $\frac{|G|}{|\orb(a)|} = |\stab(a)|$, so $|\orb(a)|$ divides $|G|$, meaning $n - 1$ divides $n$.  Thus $n = 2$.
    
    \section{Problem 6}
    Suppose that $G$ is a group of order 203 and that $H$ is a normal subgroup of index 29.  Prove that $G$ must be abelian.
        
        \subsection{Solution}
        The prime factorization of 203 is $203 = 7 \cdot 29$.  Since there is a normal subgroup of index 29 and order 7, $n_7 = 1$ (without knowing $H$ is normal, we'd only have $n_7 \in \{1, 29\}$ by Sylow's 3rd Theorem).  By Sylow's 3rd Theorem, $n_{29} \equiv 1$ (mod $29$) and $n_{29} \mid 7$.  This requires that $n_{29} = 1$.  If we let $p = 7$ and $q = 29$, we now have the same situation as in \#1, where $|G| = pq$ and $n_p, n_q = 1$.  That proof implies $G$ must be cyclic, which implies $G$ is abelian.
    
    \section{Problem 7}
    Let $G$ be a simple group with $|G| = 168$.  Determine the number of elements of order 7 in $G$.
        
        \subsection{Solution}
        The prime factorization of $|G|$ is $168 = 2^3 \cdot 3 \cdot 7$.  By Sylow's 3rd Theorem, $n_7 \in \{1, 8\}$.  Since $G$ is simple, $n_7 = 1$ is impossible because that would imply there is a normal subgroup of order 7.  Thus, $n_7 = 8$, so there are $8(7 - 1) = 48$ elements of order 7 in $G$.  Note the $7 - 1$ is because I'm not counting the identity in each subgroup of order 7, which always has order 1.
    
\end{document}
