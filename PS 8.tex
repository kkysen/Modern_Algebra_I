\documentclass[fleqn]{article}
\usepackage[margin=1in]{geometry}
\usepackage[utf8]{inputenc}
\usepackage{ulem}
\usepackage{mathtools}
\usepackage{amsmath}
\usepackage{amsthm}
\usepackage{amssymb}
\usepackage{fancyvrb}
\usepackage{cleveref}
\usepackage{centernot}
\usepackage{mathabx}

\title{
Math GU4041, Fall 2019 \\
Modern Algebra I, Prof.\ Kyler Siegel \\
Problem Set 8
}
\author{Khyber Sen}
\date{11/22/2019}

\DeclareMathOperator{\dom}{dom}
\DeclareMathOperator{\im}{im}
\DeclareMathOperator{\orb}{orb}
\DeclareMathOperator{\stab}{stab}
\DeclareMathOperator{\Cl}{Cl}
\DeclareMathOperator{\Aut}{Aut}
\DeclareMathOperator{\Inn}{Inn}

\setcounter{secnumdepth}{0}

\begin{document}
    
    \maketitle
    
    \section{Problem 1}
        
        \subsection{(1)}
        Let $G$ be a group.  By definition, an \textit{automorphism} of $G$ is an isomorphism from $G$ to itself.  We denote by $\Aut(G)$ the set of automorphisms of $G$.  Prove that $\Aut(G)$ is a group under composition.
            
            \subsubsection{Solution}
            First, $\Aut(G)$ is closed under composition because the composition of isomorphisms is an isomorphism.  Second, composition is already associative for all functions.  Third, isomorphisms are bijective, so they have inverses that are also isomorphisms.  Fourth, the identity is just the identity function.
            
        \subsection{(2)}
        Let $p$ be a prime.  Prove that the automorphism group $\Aut(\mathbb{Z}/p\mathbb{Z})$ is isomorphic to $C_{p - 1}$.
            
            \subsubsection{Solution}
            First we'll prove $\Aut(\mathbb{Z}/n\mathbb{Z}) \cong (\mathbb{Z}/n\mathbb{Z})^\times$ $\forall$ $n \in \mathbb{N}$.  We already know $(\mathbb{Z}/n\mathbb{Z})^\times \cong \mathbb{Z}/(n - 1)\mathbb{Z}$ when $n$ is prime, and then $\mathbb{Z}/(n - 1)\mathbb{Z} \cong C_{n - 1}$.  Thus, we'll have shown $\Aut(\mathbb{Z}/p\mathbb{Z}) \cong C_{p - 1}$ when $p$ is prime.
            
            To show $\Aut(\mathbb{Z}/n\mathbb{Z}) \cong (\mathbb{Z}/n\mathbb{Z})^\times$, we'll construct an isomorphism $f: (\mathbb{Z}/n\mathbb{Z})^\times \to \Aut(\mathbb{Z}/n\mathbb{Z})$.  Let $f = a \mapsto x \mapsto ax$.  Note that $ax = a^x$ in the sense that the group operation for $\mathbb{Z}/n\mathbb{Z}$ is addition, so exponentiation becomes multiplication.  Thus, $ax$ is generated by $a$.  Therefore, $\im(f(a)) = \langle a \rangle$.  Since $a \in (\mathbb{Z}/n\mathbb{Z})^\times$, $\gcd(a, n) = 1$, which means $a$ is a generator for $\mathbb{Z}/n\mathbb{Z}$.  Thus, $\langle a \rangle = \mathbb{Z}/n\mathbb{Z}$.  This means $f(a)$ is surjective.  It's also injective because $f(a)(x) = e = 0 \implies ax = 0 \implies x = 0 = e$, so $\ker(f(a)) = \{e\}$.  It's also a homomorphism because $f(a)(x + y) = a(x + y) = ax + ay = f(a)(x) + f(a)(y)$.  Thus $f(a)$ is an isomorphism, meaning $f$ is well-defined.
            
            To show $f$ itself (not $f(a)$) is a homomorphism,
            \begin{align}
                f(a)f(b) 
                    &= f(a) \circ f(b) \\
                    &= x \mapsto (f(a) \circ f(b))(x) \\
                    &= x \mapsto f(a)(f(b)(x)) \\
                    &= x \mapsto f(a)(bx) \\
                    &= x \mapsto a(bx) \\
                    &= x \mapsto (ab)x \\
                    &= x \mapsto f(ab)(x) \\
                    &= f(ab)
            \end{align}
            
            To show $f$ is bijective, we'll construct an inverse $g = h \mapsto h(1)$.  Since $h$ is an isomorphism, it's homomorphic, so $h(x) = \prod\limits_{i = 1}^x h(1) = h(1)x$.  Thus, $h$ is of the form $x \mapsto h(1)x$.  This means $h(1)$ is the same as the $a$ in $f$.  More explicitly, $f(g(h)) = f(h(1)) = x \mapsto h(1)x = h$, and $g(f(a)) = g(x \mapsto ax) = a$, so $f \circ g = g \circ f = id$, so $f$ is bijective.
            
            Since $f$ is well-defined, bijective, and homomorphic, it's an isomorphism, so $\Aut(\mathbb{Z}/n\mathbb{Z}) \cong (\mathbb{Z}/n\mathbb{Z})^\times$.  Following the rest of the proof laid out at the beginning, this implies $\Aut(\mathbb{Z}/p\mathbb{Z}) \cong C_{p - 1}$ when $p$ is prime.
        
        \subsection{(3)}
        For any prime $p$ and positive integer $n \in \mathbb{Z}^+$, determine the order of the group of automorphisms of $\mathbb{Z}/(p^n\mathbb{Z})$.
            
            \subsubsection{Solution}
            $\Aut(\mathbb{Z}/(p^n\mathbb{Z})) \cong (\mathbb{Z}/(p^n\mathbb{Z}))^\times$ from above, so
            \begin{align}
                |\Aut(\mathbb{Z}/(p^n\mathbb{Z}))| 
                    &= |(\mathbb{Z}/(p^n\mathbb{Z}))^\times| \\
                    &= p^n \prod\limits_{q \mid p^n, \thickspace q: \text{prime}} \left(1 - \frac{1}{q}\right) \\
                    &= p^n \left(1 - \frac{1}{p}\right)
            \end{align}
    
    \section{Problem 2}
        
        \subsection{Solution}
        I'm going to prove (1) and (2) simultaneously by proving a more generic version: $P \subseteq G$ is the unique Sylow $p$-subgroup of prime order $p$ and no factor of $|G|$ other than 1 divides $p - 1$ $\implies P \subseteq Z(G)$.
        
        Consider the conjugation action of $G$ on $P$.  This corresponds to a homomorphism $f: G \to \Aut(P)$ in the same way that it corresponds to a homomorphism $G \to S_P$.  If $f: G \to P \to P = g \mapsto x \mapsto gxg^{-1}$ is the conjugation action, then $f(g): P \to P$ is an automorphism.  To show $f(g)$ is a homomorphism, 
        \begin{align}
            f(g)(xy) 
                &= g(xy)g^{-1} \\
                &= gxeyg^{-1} \\
                &= gxg^{-1}gyg^{-1} \\
                &= (gxg^{-1})(gyg^{-1}) \\
                &= f(g)(x)f(g)(y)
        \end{align}.
        To show $f(g)$ is bijective, we'll construct an inverse $h(g): P \to P = x \mapsto g^{-1}xg$.  Then
        \begin{align}
            f(g)(h(g)(x)) 
                &= f(g)(g^{-1}xg) \\
                &= gg^{-1}xgg^{-1} \\
                &= x \\
            f(g) \circ h(g) &= id
        \end{align}
        The same proof works to show $h(g) \circ f(g) = id$, so then $f(g)$ is bijective, and since it's also homomorphic, it's isomorphic.  Thus, $f(g) \in \Aut(P)$, and therefore, $f: G \to \Aut(P)$.
        
        By \#1, $\Aut(P) \cong (\mathbb{Z}/p\mathbb{Z})^\times \cong \mathbb{Z}/(p - 1)\mathbb{Z}$.  By the 1st Isomorphism Theorem, $G/\ker(f) \cong \im(f) \subseteq \Aut(P)$.  Thus, by Lagrange's Theorem, $|G/\ker(f)| \mid |\Aut(P)| = p - 1$.  Since no factor of $|G|$ other than 1 divides $p - 1$, this means $|G/\ker(f)| = 1$, so $\ker(f) = G$.  Thus, $f(g) = id = x \mapsto x$ $\forall g \in G$.  Since $f(g) = x \mapsto gxg^{-1}$ by definition, $gxg^{-1} = x$ $\forall$ $x \in P$, $g \in G$.  Thus, $\forall$ $x \in P$, $x \in Z(G) = \{x \in G \mid gxg^{-1} = x$ $\forall$ $g \in G\}$, meaning $P \subseteq G$, i.e.\ $P$ is contained in the center of $G$.
        
        \subsection{(1)}
        Let $G$ be a group of order 231, and let $P \subset G$ be a Sylow 11-subgroup.  Prove that $P$ is contained in the center of $G$.
            
            \subsubsection{Solution}
            $|G| = 231 = 3 \cdot 7 \cdot 11$.  By Sylow's 3rd Theorem, $n_{11} \equiv 1$ (mod $11$) and $n_{11} \mid 21$, which implies $n_{11} = 1$, meaning $P$ is the unique Sylow 11-subgroup of order $p = 11$.  Since 3, 7, 11 don't divide $p - 1 = 10$, we can use the above general theorem to conclude $P \subseteq Z(G)$.
        
        \subsection{(2)}
        Let $G$ be a group of order 203, and let $H$ be a normal subgroup of order 7.  Prove that $H$ is contained in the center of $G$.
        
            \subsubsection{Solution}
            $|G| = 203 = 7 \cdot 29$.  $H \unlhd G$, so $H$ is the unique Sylow 7-subgroup of order $p = 7$.  Since 7 and 29 don't divide $p - 1 = 6$, we can use the above general theorem to conclude $H \subseteq Z(G)$.
    
    \section{Problem 3}
        
        \subsection{Solution}
        I'm going to prove (1) and (2) simultaneously by proving the more general theorem, that if $G$ is an abelian group of prime-power order $p^k$, then $G$ is isomorphic to the internal direct product of cyclic groups of prime-power order, i.e.\ $G \cong \bigtimes\limits_{i = 1}^m \mathbb{Z}/(p^{n_i})\mathbb{Z}$ where $\sum\limits_{i = 1}^m n_i = k$.  We can do this by proving a lemma very similar to one in class, that $\exists$ a subgroup $M \subseteq G$ and $g \in G$ s.t.\ $G \cong \langle g \rangle \times M$.  Applying this lemma repeatedly on $M$, starting with $M = G$, we can show $G \cong \bigtimes\limits_{i = 1}^m \langle g_i \rangle$.  The $\langle g_i \rangle$ are cyclic, so $\langle g_i \rangle \cong \mathbb{Z}/|g_i|\mathbb{Z}$.  Since $|g_i| \mid |G| = p^k$, the $|g_i|$ are prime powers that multiply to $p^k$, meaning $\langle g_i \rangle \cong \mathbb{Z}/(p^{n_i})\mathbb{Z}$.
        
        To prove the lemma, we can use induction on $k$.  For the base where $k = 1$, $G \cong \langle g \rangle \times \langle e \rangle$, which we know because $G$ has prime order, so it's cyclic, so it's generated by a generator $g$.
        
        Now for the inductive hypothesis we assume the theorem and lemma are true for groups of prime-power order $p^n$ where $n < k$.  Let $g \in G$ be an element of maximal order $p^m$.  If $m = k$, then $\langle g \rangle = G$, so we just have $G \cong \langle g \rangle \times \langle e \rangle$, so we're done.
        
        If $m < k$, then let $h \in G \setminus \langle g \rangle$ be an element of minimal order.  To show $|h| = p$, we know the order is some power of $p$, i.e.\ $|h| = p^n$, but we want to show $n = 1$.  If $n > 1$, then $|h^{p^n}| < p^n = |h|$, contradicting the minimality of the order of $h$.  We also know $n \neq 0$, because $p^0 = e \in \langle g \rangle$, but $h \notin \langle g \rangle$ by definition.
        
        Now, let $\bar{G} = G/\langle h \rangle$ and let the bar denote a coset of $\langle h \rangle$ in $G$, i.e.\ $\bar{x} = x \langle h \rangle$.  We want to show that $|\bar{g}| = p^m$ just like $g$.  $|\bar{g}| = p^{m - n} \implies \bar{g}^{p^{m - n}} = \bar{e} \implies g^{p^{m - n}} \in \langle h \rangle \implies g^{p^{m - n}} \in \langle g \rangle \cap \langle h \rangle = \{e\} \implies |g| = p^{m - n}$.  If $n > 0$, then $|g| \neq p^m$, which is a contradiction, so $n = 0$, meaning $|\bar{g}| = p^m$, too, i.e.\ $\bar{g}$ has maximal order in $\bar{G}$.  $|\bar{G}| = \frac{|G|}{|\langle h \rangle|} = \frac{p^k}{p} = p^{k - 1}$.  Thus, $\exists$ $\bar{M} \subseteq \bar{G}$ s.t.\ $\bar{G} \cong \langle \bar{g} \rangle \times \bar{M}$ by applying the induction hypothesis, since $|\bar{G}| = p^{k - 1}$ and $k - 1 < k$.  Now let $M = \pi^{-1}(\bar{M}) = \{x \in G \mid \bar{x} \in \bar{M}\}$.
        
        Now we can show $G = \langle g \rangle \times M$ and we'll be done.  Since $x \in \langle g \rangle \cap M \implies \bar{x} \in \langle \bar{g} \rangle \cap \bar{M} = \{\bar{e}\} \implies x \in \langle g \rangle \cap \langle h \rangle = \{e\} \implies \langle g \rangle \cap M = \{e\}$.  We also know that $|\langle g \rangle M| = \frac{|\langle g \rangle |M|}{|\langle g \rangle \cap M|} = \frac{|g| |\bar{M}| p}{1} = |\bar{g}| |\bar{M}| p = |\bar{G}| p = |G|$, meaning $G = \langle g \rangle M$, which implies that $G = \langle g \rangle \times M$ since subgroups of abelian groups are normal.
        
        \pagebreak
        
        \subsection{(1)}
        Let $G$ be a group of order $p^2$, where $p$ is a prime.  Prove that $G$ is isomorphic to either $\mathbb{Z}/(p^2\mathbb{Z})$ or $\mathbb{Z}/(p\mathbb{Z}) \times \mathbb{Z}/(p\mathbb{Z})$.  \textit{Note: You should not invoke the classification of finite abelian groups.}
            
            \subsubsection{Solution}
            Using the general theorem proved above, we know $G$ is isomorphic to a direct product of cyclic groups of prime-power order whose powers sum to 2.  Thus, the only possibilities of these powers are $2$ and $1 + 1$, corresponding to $\mathbb{Z}/p^2\mathbb{Z}$ and $\mathbb{Z}/p\mathbb{Z} \times \mathbb{Z}/p\mathbb{Z}$.  In particular, if $G$ is cyclic, then $G \cong \mathbb{Z}/(p^2\mathbb{Z})$, and $G$ is not, then $G \cong (\mathbb{Z}/p\mathbb{Z})^2$.
        
        \subsection{(2)}
        Similarly, let $G$ be an \textit{abelian} group of order $p^3$, where $p$ is a prime.  Prove that $G$ is isomorphic to a direct product of cyclic groups.
        
            \subsubsection{Solution}
            Using the general theorem proved above, we just apply it for the $k = 3$ case.
    
    \section{Problem 4}
    Let $G$ be a finite abelian group, and let $n$ be a positive divisor of $|G|$.  Prove that $G$ has a subgroup of order $n$.  \textit{Note: You should not invoke the classification of finite abelian groups.}
        
        \subsection{Solution}
        We can prove this by induction on $|G|$.  The base case is trivial because $|G| = 1$ has only one factor, 1, and $\{e\} \subseteq \{e\}$.  Now for the inductive hypothesis, assume for all abelian groups $H$ of order $|H| < |G|$, for all factors $d \mid |H|$, $H$ has a subgroup of order $d$.  Now we have to prove it for $G$.
        
        Let $n \mid |G|$ be a factor of $|G|$.  $n = mp$ for some prime $p$ (any prime of power 1).  By Cauchy's Theorem, there exists a subgroup $H \subseteq G$ of order $p$.  Since $G$ is abelian, $H$ is normal, so we can consider the quotient group $G/H$.  Since $|G/H| < |G|$, we can apply the inductive hypothesis to it.  $|G/H| = \frac{|G|}{|H|} = \frac{|G|}{p}$.  $m \mid \frac{|G|}{p}$, so by the inductive hypothesis, $G/H$ has a subgroup of order $m$.  By the 4th Isomorphism Theorem, there exists a subgroup $K$ of $G$ containing $H$, i.e. $H \subseteq K \subseteq G$, such that $K/H \subseteq G/H$, where $K/H$ is the subgroup of $G/H$ of order $m$.  Thus, $|K/H| = m \implies \frac{|K|}{|H|} = m \implies |K| = m|H| = mp = n$.  Thus, $N$ is the subgroup of $G$ of order $n$, so the induction is complete.
    
    \section{Problem 5}
    For a group $G$, an automorphism is called \textit{inner} if it is given by conjugation by some element $g \in G$.  The inner automorphisms of $G$ form a subgroup $\Inn(G) \subseteq \Aut(G)$.  Prove that every automorphism of $S_n$ is inner for $n \geq 2$ and $n \neq 6$.  \textit{Hint: Follow the outline in Dummit and Foote $\S4.4$ Problem \#18}
        
        \subsection{Solution}
        Following the outline in Dummit and Foote: \\
        $ $ \\
        \noindent
        This exercise shows that for $n \neq 6$ every automorphism of $S_n$ is inner.  Fix an integer $n \geq 2$ with $n \neq 6$.
        \begin{itemize}
            \item (a) Prove that the automorphism group of a group $G$ permutes the conjugacy classes of $G$, i.e., for each $\sigma \in \Aut(G)$ and each conjugacy class $K$ of $G$ the set $\sigma(K)$ is also a conjugacy class of $G$.
            \item (d) Let $K$ be the conjugacy class of transpositions in $S_n$ and let $K'$ be the conjugacy class of any element of order 2 in $S_n$ that is not a transposition.  Prove that $|K| = |K'|$.  Deduce that any automorphism of $S_n$ sends transpositions to transpositions.
            \item (c) Prove that for each $\sigma \in \Aut(S_n)$, $\sigma = (1 i) \mapsto (a b_i)$, $i \in 2..n$.
            \item (d) Show that $\{(1, i)\}_{i = 1}^n$ generate $S_n$ and deduce that any automorphism of $S_n$ is uniquely determined by its action on these elements.  Use (c) to show that $S_n$ has at most $n!$ automorphisms and conclude that $\Aut(S_n) = \Inn(S_n)$ for $n \neq 6$.
        \end{itemize}
        
        (a) We first want to prove that $\Aut(G)$ permutes the conjugacy classes of $G$, i.e.\ $\sigma \in \Aut(G), K = \Cl(a \in G) \implies \sigma(K) = \Cl(b)$ for some $b \in G$, i.e.\ $\sigma(K)$ is also a conjugacy class of $G$.  Note below that $\sigma$ is a homomorphism since $\sigma \in \Aut(G)$, so we can expand $\sigma$ by applying the homomorphic properties.
        \begin{align}
            \sigma(\Cl(a)) &= \sigma(\{gag^{-1} \mid g \in G\}) \\
                &= \{\sigma(gag^{-1}) \mid g \in G\} \\
                &= \{\sigma(g)\sigma(a)\sigma(g^{-1}) \mid g \in G\} \\
                &= \{\sigma(g)\sigma(a)\sigma(g)^{-1} \mid g \in G\} \\
                &= \{\sigma(g)\sigma(a)\sigma(g)^{-1} \mid \sigma(g) \in G\} \\
                &= \Cl(\sigma(a))
        \end{align}
        Since $\sigma(\Cl(a)) = \Cl(\sigma(a))$, it's still a conjugacy class.
        
        (b) Now we want to show that if $K$ is the conjugacy class of transpositions in $S_n$ and $K' = \Cl(\sigma)$ where $|\sigma| = 2$ but $\sigma$ is not a transposition, then $|K| \neq |K'|$.  Note that this obviously doesn't apply to $S_1$, which has no transpositions, but that's a trivial case.  $|K| = \binom{n}{2} = \frac{n(n - 1)}{2}$.  $K'$ is the conjugacy class of $k > 1$ transpositions, so $|K'| = \frac{n!}{2^k(n - 2k)!}$.  For $n = 6$, we can have $|K| = |K'|$, but for $n \neq 6$, this is impossible.
        
        (c) Now we want to prove that $\forall$ $\sigma \in \Aut(S_n)$, $\sigma(1 i) = (a b_i)$ $\forall$ $i \in 2..n$, where $a, b_i \in 1..n$.  $\sigma((1 i)) = (\sigma(1) \sigma(i))$, so this is true if we just let $a = \sigma(1)$ and $b_i = \sigma(i)$ $\forall$ $i \in 2..n$.
        
        (d) Now we want to show that $\{(1, i)\}_{i = 2}^n$ generates all of $S_n$.  Every permutation in $S_n$ is the product of transpositions, and every transposition $(ab) = (1a)(1b)(1a)^{-1}$, so these $(1i)$ permutations generate all of $S_n$, and they do so by conjugation.  Any automorphism of $S_n$ is uniquely determined by its action on $S_n$, and since the $(1 i)$ generate $S_n$, the action of the automorphism on the $(1 i)$ uniquely determines it.  Since there are $n - 1$ of these $(1 i)$ transpositions, and after determining each of them, we have one less number to choose for the next, since the one number they move in common is now fixed.  In other words, by (c), an automorphism $\sigma$ can send $(1 2) \mapsto (a, b_2)$.  There are $n$ choices for $a$, and then $n - 1$ choices for $b_2$.  Then $(1 3) \mapsto (a, b_3)$, so there are $n - 2$ choices for $b_3$.  If we keep going like this, we get that there are $n!$ choices for $\sigma$, so there are at most $n!$ automorphisms.  Therefore, $|\Aut(S_n)| \leq n!$.  We also know that $\Inn(S_n) \cong S_n/Z(S_n)$, so $|\Inn(S_n)| = |S_n/Z(S_n)| = \frac{|S_n|}{|Z(S_n)|} = \frac{n!}{|Z(S_n)|}$.  Thus, if $S_n$ has a trivial center, then $|\Inn(S_n)| = n!$.  $Z(S_n)$ is trivial for $n \neq 2$, but for $n = 2$ it's trivial to show that $\Inn(S_2) = \Aut(S_2)$.  Thus, since $\Inn(S_n) \subseteq \Aut(S_n)$, this implies $\Inn(S_n) = \Aut(S_n)$, meaning every automorphism is inner for $n \neq 6$.
    
\end{document}
