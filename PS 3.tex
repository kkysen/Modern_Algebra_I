\documentclass[fleqn]{article}
\usepackage[margin=1in]{geometry}
\usepackage[utf8]{inputenc}
\usepackage{ulem}
\usepackage{mathtools}
\usepackage{amsmath}
\usepackage{amsthm}
\usepackage{amssymb}
\usepackage{fancyvrb}

\title{
Math GU4041, Fall 2019 \\
Modern Algebra I, Prof.\ Kyler Siegel \\
Problem Set 3
}
\author{Khyber Sen}
\date{9/27/2019}

\DeclareMathOperator{\SO}{SO}
\DeclareMathOperator{\rot}{rot}
\DeclareMathOperator{\refl}{ref}

\setcounter{secnumdepth}{0}

\begin{document}
    
    \maketitle
    
    \section{Problem 1}
    Describe all of the homomorphisms from $\mathbb{Z}/(12\mathbb{Z})$ to $C_{12}$.  How many are there?  How many of them are isomorphisms?
        
        \subsection{Solution}
        There are 12 homomorphisms from $\mathbb{Z}/(12\mathbb{Z})$ to $C_{12}$, which map a generator in $\mathbb{Z}/12\mathbb{Z}$ to an element in $C_{12}$.  Once a generator has been mapped, all the elements in the group generated by it (all of them), which are $g^k$ can be mapped to $C_{12}$ by following the homomorphism definition.  If $f$ is the homomorphism, $f(g^k) = f(g)^k$.  It doesn't need to be surjective, so we don't need to worry about $f(g)^k$, and we know $g^k$ is all of $\mathbb{Z}/(12\mathbb{Z})$ to $C_{12}$, so it's fully defined, and since we basically defined it as a homomorphism, it must be one.
        
        These homomorphisms are only isomorphism when they map a generator to another generator, so that as each of the generators generates the whole group, there is a one-to-one onto correspondence, i.e.\ a bijective one, meaning the homomorphism is an isomorphism.  The generators of $\mathbb{Z}/12\mathbb{Z}$ are the elements relatively prime to 12, i.e.\ $(\mathbb{Z}/12\mathbb{Z}) = \{1, 5, 7, 11\}$, so there are 4 generators and thus 4 isomorphisms.
    
    \section{Problem 2}
    Let $G$ be a group and let $x \in G$ be an element.  Prove that for any $k \in \mathbb{Z}^+$, we have $\left|x^k\right| = \frac{|x|}{\gcd(|x|, k)}$.
        
        \subsection{Solution}
        The order of an element of a group is the smallest power of the element that is the identity.  Thus, we have to show that $(x^k)^{\frac{|x|}{\gcd(|x|, k)}} = e$ and that $\frac{|x|}{\gcd(|x|, k)}$ is the lowest power that satisfies this.  First, $(x^k)^{\frac{|x|}{\gcd(|x|, k)}} = (x^{|x|})^{\frac{k}{\gcd(|x|, k)}} = e^{\frac{k}{\gcd(|x|, k)}} = e$.  Then, let $n$ be the lowest power of $x^k$ that is the identity, i.e.\ $n = \left|x^k\right|$.  Thus, $(x^k)^n = a^{kn} = e$, meaning $kn$ is divisible by $|x|$.  Thus, $n \frac{k}{\gcd(|x|, k)}$ must also be divisible by $\frac{|x|}{\gcd(|x|, k)}$.  Since $\gcd(\frac{k}{\gcd(|x|, k)}, \frac{|x|}{\gcd(|x|, k)}) = 1$, we know $n$ is divisible by $\frac{|x|}{\gcd(|x|, k)}$.  Therefore, $\frac{|x|}{\gcd(|x|, k)} \leq n$, so $\frac{|x|}{\gcd(|x|, k)}$ must be the lowest possible power of $x^k$ that is the identity.
    
    \section{Problem 3}
    Recall that we defined the dihedral group $D_{2 \cdot 5}$ (a.k.a.\ $D_{10}$) to be the subgroup of $O(2)$ consisting of those orthogonal matrices that preserve the regular pentagon.  We will label the vertices in counterclockwise order as $v_1, v_2, v_3, v_4, v_5$, and we assume that $v_1 = (1, 0)$.
        
        \subsection{(1)}
        For each element of $D_{2 \cdot 5}$, write its representation as a permutation in $S_5$.  You may use for example the notation $\begin{pmatrix}
            1 & 2 & 3 & 4 & 5 \\
            2 & 3 & 4 & 5 & 1
        \end{pmatrix}$ as a shorthand for the permutation $v_1 \mapsto v_2, v_2 \mapsto v_3, v_3 \mapsto v_4, v_4 \mapsto v_5, v_5 \mapsto v_1$.
            
            \subsubsection{Solution}
            Instead of writing out the full mapping of each symmetry as a permutation, I'm writing the some of the permutations as functions $\{1..5\} \to \{1..5\}$ when simpler.  I'm also writing $[a]_n$ as the equivalence class of $\equiv a$ (mod $n$).
            
            There are 5 rotations and 5 reflections.  The rotations are $\{[n]_5 \mapsto [n + k]_5 \mid k \in \{1..5\}\}$, i.e.\ the vertices are just moved to the right by some amount $k$, wrapping around.  The reflections are a bit more complex, so I'll write them out fully:
            \begin{enumerate}
                
                \item 
                about 1: $\begin{pmatrix}
                    1 & 2 & 3 & 4 & 5 \\
                    1 & 5 & 4 & 3 & 2
                \end{pmatrix}$
                
                \item 
                about 2: $\begin{pmatrix}
                    1 & 2 & 3 & 4 & 5 \\
                    3 & 2 & 1 & 5 & 4
                \end{pmatrix}$
                
                \item 
                about 3: $\begin{pmatrix}
                    1 & 2 & 3 & 4 & 5 \\
                    5 & 4 & 3 & 2 & 1
                \end{pmatrix}$
                
                \item 
                about 4: $\begin{pmatrix}
                    1 & 2 & 3 & 4 & 5 \\
                    2 & 1 & 5 & 4 & 3
                \end{pmatrix}$
                
                \item 
                about 5: $\begin{pmatrix}
                    1 & 2 & 3 & 4 & 5 \\
                    4 & 3 & 2 & 1 & 5
                \end{pmatrix}$
                
            \end{enumerate}
        
        \subsection{(2)}
        Determine the order of each element of $D_{2 \cdot 5}$.
            
            \subsubsection{Solution}
            For each of the reflections, applying them twice gives the identity, so they're all of order 2.  For the rotations, they're isomorphic to $\mathbb{Z}/5\mathbb{Z}$ (since they're cyclic and all cyclic groups are isomorphic to a $\mathbb{Z}/n\mathbb{Z}$ of the same order; see middle of \#5).  Since 5 is prime, all the elements are generators, meaning their orders are all 5, except for the identity, which is just order 1.
        
        \subsection{(3)}
        Determine all numbers which appear as the index of some subgroup of $D_{2 \cdot 5}$.  \textit{Note: you should justify your answer.  Keep in mind that for this type of problem, Lagrange's Theorem gives only a necessary condition.}
            
            \subsubsection{Solution}
            By Lagrange's Theorem, the indices of the subgroups must be factors of $|D_{2 \cdot 5}| = 10$: 1, 2, 5, 10.  These are only the possible indices, so we need to check them.  We'll actually just check the orders of the subgroups, and then divide 10 by them.  1 always works as the subgroups with the identity.  10 also works as the group itself.  5 also works since the rotations are isomorphic to $\mathbb{Z}/5\mathbb{Z}$.  2 also works for the group $\{e, r\}$, where $e$ is the identity and $r$ is any one of the reflections, which works since $ee = e$, $er = r$, $re = r$, and $rr = e$ since it has order 2.  Therefore, all the possible indices are actually indices, so the subgroups orders are the same as the indices: 1, 2, 5, 10.
        
        \subsection{(4)}
        Is $D_{2 \cdot 5}$ abelian?
            
            \subsubsection{Solution}
            $D_{2 \cdot 5}$ isn't abelian.  As a counterexample, consider $rot_1$ (rotating by 1) and $\refl_3$ (rotating about 3).  $\rot_1(\refl_3((1, ..., 5))) = \rot_1((5, ..., 1)) = (1, ...)$.  But $\refl_3(\rot_1((1, ..., 5)) = \refl_3((5, ..., 4)) = (4, ...) \neq (1, ...)$.
            
        \subsection{(5)}
        Pick a subgroup of order 2, and write out the corresponding partition of $D_{2 \cdot 5}$ into cosets.
            
            \subsubsection{Solution}
            Let $H = \{e, \refl_1\}$ be the subgroup.
            \begin{align}
                \rot_0 \{e, \refl_1\} &= \{\rot_0, \refl_1\} \\
                \rot_1 \{e, \refl_1\} &= \{\rot_1, \refl_4\} \\
                \rot_2 \{e, \refl_1\} &= \{\rot_2, \refl_2\} \\
                \rot_3 \{e, \refl_1\} &= \{\rot_3, \refl_3\} \\
                \rot_4 \{e, \refl_1\} &= \{\rot_4, \refl_5\}
            \end{align}
            Thus, the partition is $\{\{\rot_0, \refl_1\}, \{\rot_1, \refl_4\}, \{\rot_2, \refl_2\}, \{\rot_3, \refl_3\}, \{\rot_4, \refl_5\}\}$.  Although I only multiplied the subgroup by the rotations, the partition is already complete, so multiplying by the reflections will only yield the same cosets.
        
    \section{Problem 4}
    Suppose that $G$ and $H$ are finite groups of the same order which are isomorphic.  \textit{Note: For this problem, please work directly with the definitions of cyclic group, abelian group, and isomorphism, without invoking any results we stated in class.}
        
        \subsection{(1)}
        Assume that $G$ is a cyclic group.  Prove that $H$ is also a cyclic group.
            
            \subsubsection{Solution}
            $G$ is cyclic $\implies$ $\exists$ $g \in G$ s.t.\ $\langle g \rangle = G \implies G = \{g^k \mid k \in \mathbb{N}\}$.  Since $G$ is isomorphic to $H$, $\exists$ a bijection $f: G \to H$.  To prove $H$ is also cyclic, we need to show $H$ also has a generator, namely $f(g)$, i.e.\ $H = \{f(g)^k \mid k \in \mathbb{N}\}$.  Since an isomorphism between groups is also a homomorphism, we know $f(g)^k = f(g)f(g)...f(g) = f(gg)...f(g) = f(gg...g) = f(g^k)$.  Therefore, $\{f(g)^k \mid k \in \mathbb{N}\} = \{f(g^k) \mid k \in \mathbb{N}\} = f(\{g^k \mid k \in \mathbb{N}\}) = f(G) = f(H)$.
        
        \subsection{(2)}
        Assume that $G$ is an abelian group.  Prove that $H$ is also an abelian group.
            
            \subsubsection{Solution}
            If $G$ is abelian, then $\forall$ $a, b \in G$, $ab = ba$.  We need to show that $H$ is abelian, i.e.\ $\forall$ $c, d \in H$, $cd = dc$.  Since $G$ is isomorphic to $H$, $\exists$ a bijection $f: G \to H$.  Therefore, $\forall$ $c, d \in H$, $f^{-1}(c)f^{-1}(d) = f^{-1}(d)f^{-1}(c)$ because they're in $G$, which is abelian.  Thus, we also know that $f(f^{-1}(c)f^{-1}(d)) = f(f^{-1}(d)f^{-1}(c))$.  Since $f$ is also a homomorphism, this simplifies: $f(f^{-1}(c)f^{-1}(d)) = f(f^{-1}(d)f^{-1}(c)) \implies f(f^{-1}(c))f(f^{-1}(d)) = f(f^{-1}(d))f(f^{-1}(c)) \implies cd = dc$.  Therefore, $H$ is also abelian.
        
    \section{Problem 5}
    Show directly that the group $(\mathbb{Z}/(20\mathbb{Z}))^{\times}$ is isomorphic to a direct product of cyclic groups, i.e.\ it is isomorphic to $\mathbb{Z}/(k_1 \mathbb{Z}) \times \dots \times \mathbb{Z}/(k_n \mathbb{Z})$ for some $n \in \mathbb{Z}^+$ and $k_1, \dots, k_n \in \mathbb{Z}_{\geq 2}$.
        
        \subsection{Solution}
        First, we will show that $(\mathbb{Z}/20\mathbb{Z})^\times \cong (\mathbb{Z}/4\mathbb{Z})^\times \times (\mathbb{Z}/5\mathbb{Z})^\times$, and then we will show that $(\mathbb{Z}/4\mathbb{Z})^\times \cong \mathbb{Z}/2\mathbb{Z}$ and $(\mathbb{Z}/5\mathbb{Z})^\times \cong \mathbb{Z}/4\mathbb{Z}$.  For the first isomorphism, let 
        \begin{align}
            f&: (\mathbb{Z}/20\mathbb{Z})^\times \to (\mathbb{Z}/4\mathbb{Z})^\times \times (\mathbb{Z}/5\mathbb{Z})^\times \\
            f &= [n]_{20} \mapsto ([n]_4, [n]_5) \\
            f^{-1} &= ([a]_4, [b]_5) \mapsto [5a + 4b]_{20}
        \end{align}
        Now we will show that $f^{-1}$ is a true inverse of $f$, and thus $f$ is a bijection.
        \begin{align}
            f^{-1}(f([n]_{20})) &= f^{-1}(([n]_4, [n]_5)) \\
            &= [5n + 4n]_{20} \\
            &= [9n]_{20} \\
            &= [n]_{20} \enspace \because{} \enspace [9]_{20} \in (\mathbb{Z}/20\mathbb{Z})^\times
        \end{align}
        \begin{align}
            f(f^{-1}(([a]_4, [b]_5))) &= f([5a + 4b]_{20}) \\
            &= ([5a + 4b]_4, [5a + 4b]_5) \\
            &= ([5a]_4, [4b]_5) \\
            &= ([a]_4, [b]_4) \enspace \because{} \enspace [5]_4 = [1]_4 \in (\mathbb{Z}/4\mathbb{Z})^\times, \enspace [4]_5 \in (\mathbb{Z}/5\mathbb{Z})^\times
        \end{align}
        Now we need to show $f$ is a homomorphism, and thus is an isomorphism.
        \begin{align}
            f([ab]_{20}) &= ([ab]_4, [ab]_5) \\
            &= ([a]_4 [b]_4, [a]_5 [b]_5) \\
            &= ([a]_4, [a]_5)([b]_4, [b]_5) \\
            &= f([a]_{20}) f([b]_{20})
        \end{align}
        Thus, we have that $(\mathbb{Z}/20\mathbb{Z})^\times \cong (\mathbb{Z}/4\mathbb{Z})^\times \times (\mathbb{Z}/5\mathbb{Z})^\times$.  Now we need to show each of the multiplicative modular groups in the direct product are themselves isomorphic to additive modular groups, i.e.\ $\mathbb{Z}/n\mathbb{Z}$, where $n$ is the order of the multiplicative modular group.
        
        To do this, we'll show that any cyclic group $\langle g \rangle$ is isomorphic to $\mathbb{Z}/|g|\mathbb{Z}$.  Let
        \begin{align}
            f&: \mathbb{Z}/|g|\mathbb{Z} \to \langle g \rangle \\
            f &= [a]_{|g|} \mapsto g^a \\
            f^{-1} &= g^a \mapsto [a]_{|g|}
        \end{align}
        $f^{-1}$ is pretty trivially a true inverse, so $f$ is thus a bijection.  \\
        $f$ is also a homomorphism because $f([a + b]_{|g|}) = g^{a + b} = g^a g^b = f(a)f(b)$.  
        
        Using this, we know $(\mathbb{Z}/4\mathbb{Z})^\times \cong \mathbb{Z}/2\mathbb{Z}$ and $(\mathbb{Z}/5\mathbb{Z})^\times \cong \mathbb{Z}/4\mathbb{Z}$.  The latter is obvious since 5 is prime so therefore it's cyclic and it's order is $5 - 1 = 4$.  The former, it's order is obviously 2 since $(\mathbb{Z}/4\mathbb{Z})^\times = \{1, 3\}$, but we have to show it's cyclic since 4 isn't prime.  It's small, though, so it's easy to show it's cyclic.  $[3^3]_4 = [9]_4 = [1]_4$ and $[3^1]_4 = [3]_4$, so $3$ is a generator and thus it's cyclic.
        
        Now we just have to prove $A \cong C, B \cong D \implies A \times B \cong C \times D$.  Let $f: A \to C$ and $g: B \to D$ be isomorphisms.  Let
        \begin{align}
            h&: A \times B \to C \times D \\
            h &= (a, b) \mapsto (f(a), g(b)) \\
            h^{-1} &= (c, d) \mapsto (f^{-1}(c), f^{-1}(d)) \\
        \end{align}
        $h^{-1}$ is clearly a true inverse since it just uses $f^{-1}$ and $g^{-1}$, so thus $h$ is a bijection.  Furthermore, \begin{align}
            h((a, b)(x, y)) &= h((ax, by)) \\
            &= (f(ax), g(by)) \\
            &= (f(a)f(x), g(b)g(y)) \enspace \because{} \enspace f, g \text{ are homomorphisms}\\
            &= (f(a), g(b))(f(x), g(y)) \\
            &= h((a, b))h((x, y))
        \end{align}
        so $h$ is also a homomorphism and thus an isomorphism.
        
        Therefore, $(\mathbb{Z}/4\mathbb{Z})^\times \times (\mathbb{Z}/5\mathbb{Z})^\times \cong (\mathbb{Z}/2\mathbb{Z}) \times (\mathbb{Z}/4\mathbb{Z})$.  We also had $(\mathbb{Z}/(20\mathbb{Z}))^\times \cong (\mathbb{Z}/4\mathbb{Z})^\times \times (\mathbb{Z}/5\mathbb{Z})^\times$.  By the transitivity of isomorphisms, $(\mathbb{Z}/(20\mathbb{Z}))^\times \cong (\mathbb{Z}/2\mathbb{Z}) \times (\mathbb{Z}/4\mathbb{Z})$.
    
    \section{Problem 6}
    We define $\SO(3)$ to be the group of $3 \times 3$ orthogonal matrices whose determinant is 1.  This is the group of rotations in three-space, and you can visualize each element as a rotation about some axis by some angle.
        
        \subsection{(1)}
        Check that $\SO(3)$ satisfies the three axioms of a group.  \textit{You may take for granted that matrix multiplication is associative, as well as any standard properties of transposes and determinants.}
            
            \subsubsection{Solution}
                First, we have to show the group operation, matrix multiplication, is closed under $\SO(3)$, i.e.\ $\forall$ $A, B \in \SO(3)$, $AB \in \SO(3)$.  More specifically, we need to show $\forall$ $A, B$ s.t.\ $|A| = |B| = 1$, $|AB| = 1$.  Since the determinant of the product of matrices is the product of the determinant of the matrices, $|AB| = |A||B| = 1 \cdot 1 = 1$, so therefore the $\SO(3)$ is closed under multiplication.
                
                For associativity, matrix multiplication is already associative, so we don't need to show anything.
                
                For the identity, let $e = I_3 = \begin{bmatrix}
                    1 & 0 & 0 \\
                    0 & 1 & 0 \\
                    0 & 0 & 1
                \end{bmatrix}$.  $\forall$ $A \in \mathbb{R}^{n \times n}$, $AI_n = I_nA = A$.  Thus, when we narrow the set from $\mathbb{R}^{n \times n}$ to just $\SO(3)$, the same identity still obviously works.
                
                For inverses, a matrix $A$ has a unique inverse $\iff |A| \neq 0$.  Since all the matrices in $\SO(3)$ have determinant $1 \neq 0$, they are all invertible.
        
        \subsection{(2)}
        Prove that any reflection about a two-plane (or rather, the matrix representation of such a linear transformation) is \textit{not} included in $\SO(3)$.  \textit{Hint: What is the determinant of such a matrix?  Note that after a change of basis you can take the plane to be the xy-plane.}
            
            \subsubsection{Solution}
            For any reflection across a two-plane, we can change its basis such that it is mapped onto a reflection across the xy-plane.  Since a change of basis doesn't change the determinant of the transformation, the reflection across the two-plane will be in $\SO(3)$ iff the the reflection across xy-plane is in $\SO(3)$, i.e.\ its determinant is 1.  This matrix is $\begin{bmatrix}
                1 & 0 & 0 \\
                0 & 1 & 0 \\
                0 & 0 & -1
            \end{bmatrix}$.  Its determinant, $\begin{vmatrix}
                1 & 0 & 0 \\
                0 & 1 & 0 \\
                0 & 0 & -1
            \end{vmatrix} = 1 \begin{vmatrix}
                1 & 0 \\
                0 & -1
            \end{vmatrix} = 1(-1 - 0) = -1 \neq 0$, so therefore reflections across a two-plane are no in $\SO(3)$.
        
        \subsection{(3)}
        Show that $\SO(3)$ is non-abelian.
            
            \subsubsection{Solution}
            Since $\SO(3)$ is the group of 3D rotations, we can just show an example of two 3D rotations that don't commute: a $90^{\circ}$ rotation about the x-axis ($X$) and a $90^{\circ}$ rotation about the y-axis ($Y$), both counterclockwise.  We can show they don't commute by applying them both to the same point, $(0, 0, 1)$.  If they did commute, the point would end up in the same place both ways.  But, $XY (0, 0, 1) = X (-1, 0, 0) = (-1, 0, 0)$ and $YX (0, 0, 1) = Y (0, 1, 0) = (0, 1, 0)$.  $(-1, 0, 0) \neq (0, 1, 0)$, so therefore, 3D rotations don't commute, and thus their representations as matrices in $\SO(3)$ don't commute either, and thus $\SO(3)$ is not abelian.
        
        \subsection{(4)}
        Consider the cube in $\mathbb{R}^3$ whose set of eight vertices is $\{(i, j, k) \mid i, j, k \in \{1, -1\}\}$.  Let $H \subseteq \SO(3)$ be the subgroup consisting of those rotations which map this cube to itself setwise (where \textit{setwise} means that each point in the cube gets mapped to another point in the cube, but the individual points of the cube might get shuffled around).  What is the order of $H$?  \textit{You should provide some justification for your answer.}
            
            \subsubsection{Solution}
            Transformations that map a cube onto itself are just symmetries, and since $H$ as a subgroup of $\SO(3)$ are only rotations, $|H|$ is the number of rotational symmetries of the cube.  These cube rotations are:
            \begin{itemize}
                \item 1: identity
                \item 6: $90^{\circ}$ rotation about x, y, z axis, ccw or cw (3 axes, 2 directions)
                \item 3: $180^{\circ}$ rotation about x, y, z axis (3 axes)
                \item 8: $120^{\circ}$ rotation about corner to corner diagonals (4 diagonals, 2 directions)
                \item 6: $180^{\circ}$ rotation about axes from center of one edge to center of the other edge across from it (12 edges, 2 per edge)
            \end{itemize}
            In total, there are 24 rotational symmetries of the unit cube, so $|H| = 24$.
        
        \subsection{(5)}
        Observe that each element of $H$ determines a permutation of the set of 8 vertices.  Give an example of such a permutation that does \textit{not} arise from an element of $H$.  \textit{Note: If you wish to identity such a permutation with an element of $S_8$, you will first need to choose an ordering of the vertices.}
            
            \subsubsection{Solution}
            A permutation of the corners of a cube is just a symmetry of the cube.  These symmetries are rotations and reflections, but $\SO(3)$ is only rotations, so $H$ must also be only rotations.  Therefore, any permutation of the 8 vertices that is a reflection is not in $H$.  For example, take the reflection across the xy-plane.  The top and bottom vertices are flipped, but it's not a rotation, so it's not in $H$.
        
        \subsection{(6)}
        Similarly, each element of $H$ determines a permutation of the set of 6 faces.  Give an example of such a permutation that does \textit{not} arise from an element of $H$.
            
            \subsubsection{Solution}
            Once again, the permutations are all the symmetries, $H$ is all the rotations but not the reflections, so any reflection is a permutation that is not in $H$.  For example, take the reflection across the xy-plane.  The top and bottom are flipped, but the sides stay the same.  This is a permutation of the faces, but it's not a rotation, so it's not in $H$.
        
        
\end{document}
